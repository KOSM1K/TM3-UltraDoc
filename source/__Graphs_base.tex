\documentclass[../TM3-UltraDoc.tex]{subfiles}
\begin{document}
	\section*{База по графам}
	\addcontentsline{toc}{section}{--- База по графам ---}
	% Your content here
	
	%\begin{tcolorbox}[colframe=gray, left=5pt, right=5pt, top=5pt, bottom=5pt, boxrule=1pt]
	%\end{tcolorbox}
	\textbf{Граф} - упорядоченная пара \(G = \langle V, E \rangle \)\\
	\(V\) - множество вершин (vertices) ((вертексы))\\
	\(E\) - множество ребер (edges) ((эджи)) \\
	\small
	\(
	\\
	V \neq \emptyset \quad \textit{правда есть nullgraph, у него вершин нет, а вот на вики написано, что мн-во вершин не пустое.} \\
	E \subseteq V \times V\\
	\\
	\)
	\normalsize
	\noindent
	\(V(G)\) - "получить"\ множество вершин графа $G$\\
	\(E(G)\) - "получить"\ множество ребер графа $G$\\
	\textit{если вместо $G$ будет не граф, а что то также имеющее вершины/ребра, то такая нотация также будет работать (например с путями в графе)}\\
	\\
	\textbf{Порядок (order)} графа - колличество вершин	$ |V(G)| $ \\
	\textbf{Размер (size)} графа - колличество ребер	$ |E(G)| $ \\
	\\
	\textit{все остальное будет в блоках по темам, на этом в общем то все}\\
	\small
	\begin{tcolorbox}[colframe=gray!50!black, left=5pt, right=5pt, top=5pt, bottom=5pt, boxrule=1pt, title=\textbf{Немножечко филосовских уточнений/рассуждений:}, colback=gray!10!white]
		\begin{itemize}
		\item Таки а что такое вершина?\\
		Вершина это вершина. Смиритесь. Можете считать ее (для удобства мышления, а не для ответа преподу) как идентификатор вершины на картинке (типо как имя или номер)
		\item Таки а что такое ребро?\\
		Ребро это пара (неупорядоченная если граф неориентированный, упорядоченная если граф ориентированный) вершин. (А как же взвешенные ребра? Я думаю о том, что пара двух вершин, это база, и никуда вы от нее не денетесь. А свойств у ребра может быть много, так что я бы описал это как:\\ 
		$e \in E = \langle \langle v_1, v_2 \rangle, \langle \text{список свойств} \rangle$)
		\end{itemize}
		
	\end{tcolorbox}
	\normalsize
		
\end{document}
