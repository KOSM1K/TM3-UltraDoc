\documentclass[../TM3-UltraDoc.tex]{subfiles}
\begin{document}
	\section*{1) Directed and undirected graphs}
	\addcontentsline{toc}{section}{1) Directed and undirected graphs}
	% Your content here
	Ориентированное ребро - ребро, у которого есть направление\\
	Неориентированное ребро - ребро, у которого нет направления\\
	\\
	\noindent
	\textit{В базе уже написано, что упорядоченная/неупорядоченная пара. Надо добавить, что ориентация ребра накладывает ограничения на пути в графе (не можешь ты пройти против направления ребра), и это влияет на работу алгоритмов, и позволяет некоторым видам графов существовать впринципе (дерево, например)}\\
	\\
	\noindent
	\textbf{Ориентированный(Directed)} - граф, у которого все ребра ориентированы (имеют направление).\\
	\textbf{Неориентированный(Undirected)} - граф, у которого все ребра неориентированы (не имеют направления).\\
	\\
	\textit{даа... об этом так интересно говорить вообще без контекста...}
	\\
	\small
	\begin{tcolorbox}[colframe=gray!50!black, left=5pt, right=5pt, top=5pt, bottom=5pt, boxrule=1pt, title=\textbf{Так что контекста навалю я}, colback=gray!10!white]
		В первую очередь, направленность / ненаправленность влияет на пути, циклы, обходы, алгоритмы поиска и тп.\\
		Некоторые подвиды графов бывают только одного вида (например деревья всегда ориентированные)
	\end{tcolorbox}
	\normalsize
		
\end{document}
